\documentclass[12pt,a4paper]{article}

% Packages
\usepackage[utf8]{inputenc}
\usepackage[T1]{fontenc}
\usepackage{amsmath,amssymb}
\usepackage{geometry}
\usepackage{booktabs}
\usepackage{listings}

% Page geometry
\geometry{margin=1in}

% Title
\title{
    \vspace{-1cm}
    \textbf{Cramer-Shoup Cryptosystem}\\[0.5cm]
    \large Public Key Cryptography Laboratory\\
    Lab 5 Documentation
}
\author{Cryptography Course}
\date{\today}

\begin{document}

\maketitle

\begin{abstract}
This document describes the Cramer-Shoup cryptosystem implementation, a public-key 
encryption scheme with IND-CCA2 security. The implementation uses a 27-character 
alphabet and includes key generation, encryption with plaintext validation, and 
decryption with ciphertext validation.
\end{abstract}

\tableofcontents
\newpage

%==============================================================================
\section{Introduction}
%==============================================================================

The Cramer-Shoup cryptosystem, introduced by Cramer and Shoup in 1998, was the 
first practical public-key encryption scheme proven secure against adaptive 
chosen-ciphertext attacks (IND-CCA2) under standard cryptographic assumptions.

The cryptosystem extends ElGamal by adding a verification mechanism that prevents 
ciphertext malleability, achieving CCA2 security based on the Decisional 
Diffie-Hellman (DDH) assumption and collision-resistant hash functions.

%==============================================================================
\section{Project Features}
%==============================================================================

\subsection{Setting: 27-Character Alphabet}

\textbf{(i) Alphabet Configuration}

The system uses a 27-character alphabet:
\begin{itemize}
    \item Space encoded as 0
    \item Letters A-Z encoded as 1-26
\end{itemize}

Plaintext is converted to an integer using base-27 encoding:
\begin{equation}
m = \sum_{i=0}^{n-1} c_i \cdot 27^{n-1-i}
\end{equation}

where $c_i$ is the numeric value of the $i$-th character.

\subsection{Key Generation}

\textbf{(ii) Public and Private Key Generation}

The system generates keys with randomly selected values in the required intervals:

\begin{enumerate}
    \item Generate safe prime $p = 2q + 1$ where $q$ is also prime
    \item Select two independent generators $g_1, g_2$ of the subgroup of order $q$
    \item Randomly select private exponents $x_1, x_2, y_1, y_2, z \in [1, q-1]$
    \item Compute public values:
    \begin{align}
        c &= g_1^{x_1} \cdot g_2^{x_2} \pmod{p} \\
        d &= g_1^{y_1} \cdot g_2^{y_2} \pmod{p} \\
        h &= g_1^{z} \pmod{p}
    \end{align}
\end{enumerate}

\textbf{Public Key:} $(p, q, g_1, g_2, c, d, h)$

\textbf{Private Key:} $(x_1, x_2, y_1, y_2, z)$

\subsection{Encryption with Plaintext Validation}

\textbf{(iii) Encryption Using Public Key}

\textbf{Plaintext validation} checks:
\begin{itemize}
    \item Plaintext is non-empty
    \item Each character is space or A-Z (case insensitive)
    \item Encoded message fits in group ($m < p$)
\end{itemize}

\textbf{Encryption process:}
\begin{enumerate}
    \item Convert plaintext to uppercase and encode as integer $m$
    \item Select random $k \in \mathbb{Z}_q$
    \item Compute ciphertext:
    \begin{align}
        u_1 &= g_1^k \pmod{p} \\
        u_2 &= g_2^k \pmod{p} \\
        e &= m \cdot h^k \pmod{p} \\
        \alpha &= H(u_1, u_2, e) \\
        v &= c^k \cdot d^{k \cdot \alpha} \pmod{p}
    \end{align}
\end{enumerate}

\textbf{Output:} Ciphertext $(u_1, u_2, e, v)$ with plaintext length.

\subsection{Decryption with Ciphertext Validation}

\textbf{(iv) Decryption Using Private Key}

\textbf{Ciphertext validation} checks:
\begin{itemize}
    \item All components in range $(0, p)$
    \item Length field is positive
\end{itemize}

\textbf{Decryption process:}
\begin{enumerate}
    \item Compute $\alpha = H(u_1, u_2, e)$
    \item Verify integrity: $u_1^{x_1 + y_1 \cdot \alpha} \cdot u_2^{x_2 + y_2 \cdot \alpha} \equiv v \pmod{p}$
    \item If verification fails, reject ciphertext
    \item Recover message: $m = e \cdot (u_1^z)^{-1} \pmod{p}$
    \item Decode integer back to plaintext
\end{enumerate}

%==============================================================================
\section{Mathematical Background}
%==============================================================================

\subsection{Safe Primes}

A \textbf{safe prime} $p = 2q + 1$ where $q$ is also prime. This ensures the 
multiplicative group $\mathbb{Z}_p^*$ has a subgroup of prime order $q$ with no 
small subgroups, making the DDH assumption hold.

\subsection{Decisional Diffie-Hellman Assumption}

Given $(g, g^a, g^b)$, it is computationally infeasible to distinguish $g^{ab}$ 
from a random group element. This assumption underlies the security of 
Cramer-Shoup.

\subsection{Hash Function}

A collision-resistant hash function $H: G^3 \rightarrow \mathbb{Z}_q$ binds 
ciphertext components together for integrity verification. The implementation 
uses SHA-256.

%==============================================================================
\section{Implementation Details}
%==============================================================================

\subsection{Alphabet Encoding}

Characters are encoded as: Space=0, A=1, B=2, ..., Z=26. Messages are converted 
to integers using base-27 encoding.

\subsection{Key Generation}

Safe primes are generated using the Miller-Rabin primality test. Generators are 
found by squaring random elements in $\mathbb{Z}_p^*$. All random selections use 
cryptographically secure random number generation.

\subsection{Validation}

\textbf{Plaintext validation:} Checks for valid characters and message length.

\textbf{Ciphertext validation:} Verifies component ranges and integrity using 
the verification tag $v$.

%==============================================================================
\section{Example Execution}
%==============================================================================

The demonstration shows two cases:

\subsection{Case 1: Successful Encryption and Decryption}

\begin{lstlisting}[language=bash]
Plaintext: 'HELLO WORLD'
Encryption: Valid plaintext → Ciphertext (u1, u2, e, v)
Decryption: Valid ciphertext → 'HELLO WORLD'
Success: Decryption matches original!
\end{lstlisting}

\subsection{Case 2: Tampered Ciphertext Detection}

\begin{lstlisting}[language=bash]
Plaintext: 'SECRET MESSAGE'
Encryption: Valid plaintext → Ciphertext
Tampering: Modify ciphertext component 'e'
Decryption: Integrity verification fails
Success: Tampered ciphertext detected and rejected
\end{lstlisting}

%==============================================================================
\section{Conclusion}
%==============================================================================

The Cramer-Shoup cryptosystem provides IND-CCA2 security with built-in integrity 
verification. This implementation demonstrates all required features: 27-character 
alphabet, random key generation, encryption with plaintext validation, and 
decryption with ciphertext validation.

%==============================================================================
% References
%==============================================================================

\begin{thebibliography}{9}

\bibitem{cramer1998}
Ronald Cramer and Victor Shoup.
\textit{A Practical Public Key Cryptosystem Provably Secure Against Adaptive Chosen Ciphertext Attack}.
In Advances in Cryptology --- CRYPTO '98, LNCS 1462, pages 13--25. Springer, 1998.

\end{thebibliography}

\end{document}
